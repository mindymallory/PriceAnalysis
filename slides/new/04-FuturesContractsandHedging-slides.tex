\documentclass[aspectratio=169]{beamer}
\usetheme{Madrid}

\usepackage{graphicx}
\usepackage{booktabs}
\usepackage{hyperref}
\usepackage{array}
\usepackage{multirow}

% Purdue Colors
\definecolor{PurdueGold}{HTML}{CEB888}
\definecolor{PurdueBlack}{HTML}{000000}

% Apply Purdue color theme
\setbeamercolor{palette primary}{bg=PurdueBlack,fg=PurdueGold}
\setbeamercolor{palette secondary}{bg=PurdueGold,fg=PurdueBlack}
\setbeamercolor{palette tertiary}{bg=PurdueBlack,fg=PurdueGold}
\setbeamercolor{palette quaternary}{bg=PurdueGold,fg=PurdueBlack}
\setbeamercolor{structure}{fg=PurdueBlack}
\setbeamercolor{section in toc}{fg=PurdueBlack}
\setbeamercolor{title}{fg=PurdueGold,bg=PurdueBlack}
\setbeamercolor{frametitle}{fg=PurdueGold,bg=PurdueBlack}
\setbeamercolor{block title}{bg=PurdueBlack,fg=PurdueGold}
\setbeamercolor{block body}{bg=PurdueGold!20,fg=PurdueBlack}
\setbeamercolor{item}{fg=PurdueBlack}
\setbeamercolor{subitem}{fg=PurdueBlack}
\setbeamercolor{subsubitem}{fg=PurdueBlack}

\title{Futures and Hedging Review}
\subtitle{Chapter 4}
\author{Mindy Mallory}
\date{}

\begin{document}

\begin{frame}
\titlepage
\end{frame}

\begin{frame}{Highlights}
\begin{itemize}
    \item Review of futures contracts and how to calculate profit or loss on a trade
    \item Hedging examples from the sell side (farmer's hedge) and from the buy side (flour mill's hedge)
    \item Learn how basis risk impacts a hedge
\end{itemize}
\end{frame}

\begin{frame}{Check Your Understanding}
\begin{itemize}
    \item Can you calculate the profit or loss from a trade?
    \item Can you fill out a hedging net revenue table given prices on key dates?
\end{itemize}
\end{frame}

\begin{frame}{What is a Hedge?}
Merriam-Webster defines a \textbf{hedge} as a means of protection or defense (as against a financial loss).

\vspace{1em}

In the present context:
\begin{itemize}
    \item A hedge is the use of a derivative contract to reduce or eliminate risk in your business' profits
    \item Understanding who is hedging and why is very helpful to understanding price relationships and what drives them to move up or down
\end{itemize}
\end{frame}

%------------------------------------------------------------
% FUTURES CONTRACT REVIEW
%------------------------------------------------------------

\begin{frame}{}
\centering
\Huge Futures Contract Review
\end{frame}

\begin{frame}{What is a Futures Contract?}
A futures contract is a contract between two parties to buy and sell:
\begin{itemize}
    \item At an \textbf{agreed-to price}
    \item A \textbf{specific quantity and quality} of something
    \item At a \textbf{specific location}
\end{itemize}

\vspace{1em}

\textbf{Example: CBOT Corn Futures}
\begin{itemize}
    \item 5,000 bushels of U.S. number 2 yellow corn
    \item At specific elevators along the Illinois River, Lake Michigan, or associated canals
\end{itemize}
\end{frame}

\begin{frame}{How Futures Differ from Stocks}
\begin{itemize}
    \item Futures are traded on a centralized exchange (like the ``stock market'')
    \item Counter-parties do not know each other
    \item \textbf{Key difference:} When a trade takes place, no ownership transfer occurs
    \item It is simply a \textbf{promise to buy or sell} at a specific date in the future
    \item This is why there are many different futures contract ``months'' (e.g., December 2017 corn, March 2018 corn)
\end{itemize}
\end{frame}

\begin{frame}{Margin and Marking to Market}
\textbf{Margin:} An amount of money that acts as a performance bond
\begin{itemize}
    \item Ensures both parties can make good on the contract if held until expiration
\end{itemize}

\vspace{1em}

\textbf{Marking to Market:} Daily transfer of money from losers to winners
\begin{itemize}
    \item As price moves up: money taken from seller's margin account, put into buyer's account
    \item As price moves down: money taken from buyer's margin account, put into seller's account
    \item Ensures everyone has financial capital required to honor contract terms
\end{itemize}
\end{frame}

\begin{frame}{Futures Trading Example}
At 10am the March corn futures contract is trading at \$4.50/bu.
\begin{itemize}
    \item Trader A decides to \textbf{buy} one contract
    \item Trader B decides to \textbf{sell} one contract
\end{itemize}

\vspace{0.5em}

At 1pm both traders close their position. Price has moved up to \$4.75/bu.

\vspace{0.5em}

Since price went up by \$0.25/bu:
\begin{itemize}
    \item Longs (buyers) gain: $+\$0.25 \times 5000 = +\$1250$
    \item Shorts (sellers) lose: $-\$0.25 \times 5000 = -\$1250$
\end{itemize}
\end{frame}

\begin{frame}{Futures Trading Example}
\begin{table}
\centering
\small
\begin{tabular}{lll}
\toprule
\textbf{Time} & \textbf{Trader A} & \textbf{Trader B} \\
\midrule
10am & buy \$4.50 & sell \$4.50 \\
1pm & sell \$4.75 & buy \$4.75 \\
\midrule
profit per bu & \$4.75 - \$4.50 = +\$0.25 & \$4.50 - \$4.75 = -\$0.25 \\
profit one contract & +\$1,250 & -\$1,250 \\
\bottomrule
\end{tabular}
\end{table}
\end{frame}

\begin{frame}{Closing Futures Positions}
Very few contracts are held all the way to expiration (when transfer of ownership would take place).

\vspace{1em}

\textbf{Traders who originally bought:}
\begin{itemize}
    \item Sometime before expiration will \textbf{sell}, eliminating their obligation
    \item Getting them to a ``flat'' position
\end{itemize}

\vspace{0.5em}

\textbf{Traders who originally sold:}
\begin{itemize}
    \item Will \textbf{buy}, getting them to a ``flat'' position
\end{itemize}
\end{frame}

%------------------------------------------------------------
% DELIVERY
%------------------------------------------------------------

\begin{frame}{}
\centering
\Huge Delivery
\end{frame}

\begin{frame}{Deliverable Contracts}
CME Group grain and oilseed futures are \textbf{deliverable contracts}:
\begin{itemize}
    \item Ensures a link between the spot market and the futures market
    \item Only a few contracts traded are actually delivered on
    \item Purpose of futures market is not physical exchange, but rather \textbf{streamlined risk management} in a liquid market
\end{itemize}
\end{frame}

\begin{frame}{Regular for Delivery}
The exchange has designated large commercial grain handlers and warehouses as \textbf{regular for delivery}.

\vspace{1em}

\textbf{Delivery Territories (CME Group Rulebook):}
\begin{itemize}
    \item Chicago and Burns Harbor, Indiana Switching District
    \item Lockport-Seneca Shipping District
    \item Ottawa-Chillicothe Shipping District
    \item Peoria-Pekin Shipping District
\end{itemize}
\end{frame}

\begin{frame}{Shipping Certificates}
If someone with a short futures position wants to deliver:
\begin{itemize}
    \item Cannot just show up with 5,000 bushels in trucks
    \item Must deliver a \textbf{shipping certificate} for 5,000 bushels from a regular for delivery warehouse
    \item A certificate giving the holder the right to demand load-out of grain meeting quality standards
\end{itemize}

\vspace{1em}

\textbf{Premium Charges:}
\begin{itemize}
    \item Fees collected by warehouse for storage space
    \item Corn and soybeans: shall not exceed 0.165 cents/bu/day ($\approx$4.95 cents/bu/month)
\end{itemize}
\end{frame}

\begin{frame}{Taking Delivery}
If the long party wants to take delivery:
\begin{itemize}
    \item Must pay the full amount: Futures Price $\times$ 5,000 bushels
    \item Must pay premium charges to the warehouse to hold the shipping certificate
\end{itemize}
\end{frame}

%------------------------------------------------------------
% HEDGING EXAMPLES
%------------------------------------------------------------

\begin{frame}{}
\centering
\Huge Who Hedges and How?
\end{frame}

%------------------------------------------------------------
% FARMER HEDGING
%------------------------------------------------------------

\begin{frame}{}
\centering
\Huge Farmer's Hedge
\end{frame}

\begin{frame}{The Farmer's Situation}
Consider a corn farmer who:
\begin{itemize}
    \item Plants crop in April
    \item Waits for it to grow and mature through summer and fall
    \item Harvests in November
    \item Sells right after harvest at local grain elevator
\end{itemize}

\vspace{1em}

\textbf{The Problem:} Entire income for the year comes from the sale of this grain, and a lot can happen to price between April and November.
\end{frame}

\begin{frame}{A Grain Elevator in Royal, IL}
\centering
\includegraphics[width=0.55\textwidth]{../../assets/Grain_elevator_royal_IL.jpg}

\tiny{Source: Daniel Schwen, CC-License}
\end{frame}

\begin{frame}{Ways to Reduce Income Risk}
\textbf{1. Crop Insurance}
\begin{itemize}
    \item Purchase a policy that pays indemnity if price goes down, yield is low, or both
    \item Does not directly interact with futures market
\end{itemize}

\vspace{1em}

\textbf{2. Forward Price Contract}
\begin{itemize}
    \item Local elevators offer forward contracts to buy grain
    \item Farmer delivers grain within specific date range at agreed price
    \item Eliminates price uncertainty for farmer
    \item Elevator transfers price risk to speculators by selling futures
\end{itemize}
\end{frame}

\begin{frame}{Ways to Reduce Income Risk (cont.)}
\textbf{3. Futures Market}
\begin{itemize}
    \item Farmer goes directly to futures market to sell futures contracts
    \item Reduces but does not eliminate uncertainty (still faces basis risk)
\end{itemize}

\vspace{1em}

\textbf{4. Options Market}
\begin{itemize}
    \item Buy a put option for a premium paid upfront
    \item Put makes money if price goes down (like short futures)
    \item If price goes up, only loses original premium
    \item Maintains upside potential while protecting downside
\end{itemize}
\end{frame}

\begin{frame}{The Basis}
\textbf{On May 1st:}
\begin{itemize}
    \item Local elevator offers \$3.50 for corn
    \item May futures trading at \$3.60
    \item The \$0.10 difference is largely due to geographic distance
\end{itemize}

\vspace{1em}

\begin{block}{Basis Definition}
$$\text{Basis} = \text{Cash Price} - \text{Futures Price}$$
\end{block}

\vspace{0.5em}

Basis reflects the price distance over space between the local cash market and the futures delivery point.
\end{frame}

\begin{frame}{The Hedging Rule}
\begin{block}{To Hedge}
Take the same action in the futures market (buy or sell) that you will do in the cash market at a future date.
\end{block}

\vspace{1em}

\textbf{The farmer will:}
\begin{itemize}
    \item \textbf{Sell} corn in the cash market in November
    \item Therefore, should \textbf{sell} (December) futures to hedge
\end{itemize}
\end{frame}

\begin{frame}{Naturally Long}
The farmer is \textbf{naturally long}:
\begin{itemize}
    \item Essentially long the unharvested grain in the field
    \item ``Long'' is a financial position that gains value when price goes up and loses value when price goes down
\end{itemize}

\vspace{1em}

\textbf{To hedge:}
\begin{itemize}
    \item Take a position with the opposite profile
    \item Take a \textbf{short position} to hedge their naturally long position
\end{itemize}
\end{frame}

\begin{frame}{Example Setup}
\textbf{On May 1st:}
\begin{itemize}
    \item Cash price in Royal, IL: \$3.50
    \item December futures price: \$3.75
    \item Basis (Dec): -\$0.25
\end{itemize}

\vspace{1em}

The farmer sells December futures to hedge.

\vspace{1em}

Consider two scenarios for November:
\begin{enumerate}
    \item Dec futures price went up to \$4.00
    \item Dec futures price went down to \$3.60
\end{enumerate}
\end{frame}

\begin{frame}{Case 1: Price Up to \$4.00, Basis Unchanged}
\begin{table}
\centering
\footnotesize
\begin{tabular}{p{2cm}p{2.5cm}p{2cm}p{2.5cm}p{1.8cm}}
\toprule
\textbf{Date} & \textbf{Action} & \textbf{Cash Price} & \textbf{Dec Futures} & \textbf{Basis} \\
\midrule
May 1st & Sell Dec Futures & \$3.50 & \$3.75 & -\$0.25 \\
Nov 1st & Buy Dec Futures \& Sell Cash Corn & \$3.75 & \$4.00 & -\$0.25 \\
\midrule
\multicolumn{2}{l}{Profit Calculation:} & & & \\
\multicolumn{2}{l}{Cash:} & \$3.75 & & \\
\multicolumn{2}{l}{Futures:} & & \$3.75 - \$4.00 = -\$0.25 & \\
\multicolumn{2}{l}{\textbf{Net per bushel:}} & \multicolumn{2}{l}{\$3.75 - \$0.25 = \textbf{\$3.50}} & \\
\bottomrule
\end{tabular}
\end{table}
\end{frame}

\begin{frame}{Case 1: Key Takeaway}
With basis unchanged:
\begin{itemize}
    \item The farmer \textbf{eliminated uncertainty} over selling price
    \item By ``selling ahead'' in the futures market, price was locked in (except for basis)
    \item Farmer would have liked to sell for \$3.75 instead of \$3.50 net
    \item But by locking in, \textbf{gave up potential for upside}
\end{itemize}
\end{frame}

\begin{frame}{Case 2: Price Down to \$3.60, Basis Unchanged}
\begin{table}
\centering
\footnotesize
\begin{tabular}{p{2cm}p{2.5cm}p{2cm}p{2.5cm}p{1.8cm}}
\toprule
\textbf{Date} & \textbf{Action} & \textbf{Cash Price} & \textbf{Dec Futures} & \textbf{Basis} \\
\midrule
May 1st & Sell Dec Futures & \$3.50 & \$3.75 & -\$0.25 \\
Nov 1st & Buy Dec Futures \& Sell Cash Corn & \$3.35 & \$3.60 & -\$0.25 \\
\midrule
\multicolumn{2}{l}{Profit Calculation:} & & & \\
\multicolumn{2}{l}{Cash:} & \$3.35 & & \\
\multicolumn{2}{l}{Futures:} & & \$3.75 - \$3.60 = +\$0.15 & \\
\multicolumn{2}{l}{\textbf{Net per bushel:}} & \multicolumn{2}{l}{\$3.35 + \$0.15 = \textbf{\$3.50}} & \\
\bottomrule
\end{tabular}
\end{table}
\end{frame}

\begin{frame}{Case 2: Key Takeaway}
\textbf{The hedge protected against deteriorating prices!}

\vspace{1em}

By selling Dec futures ahead of the cash sale:
\begin{itemize}
    \item Price risk was reduced
    \item But not eliminated---still have \textbf{basis risk}
\end{itemize}
\end{frame}

\begin{frame}{Basis Risk Examples}
Now consider: futures price unchanged at \$3.75 in November

\vspace{1em}

But the basis changes:
\begin{itemize}
    \item Case 3: Basis widens to -\$0.50
    \item Case 4: Basis narrows to \$0.00
\end{itemize}
\end{frame}

\begin{frame}{Case 3: Futures Unchanged, Basis Widens to -\$0.50}
\begin{table}
\centering
\footnotesize
\begin{tabular}{p{2cm}p{2.5cm}p{2cm}p{2.5cm}p{1.8cm}}
\toprule
\textbf{Date} & \textbf{Action} & \textbf{Cash Price} & \textbf{Dec Futures} & \textbf{Basis} \\
\midrule
May 1st & Sell Dec Futures & \$3.50 & \$3.75 & -\$0.25 \\
Nov 1st & Buy Dec Futures \& Sell Cash Corn & \$3.25 & \$3.75 & -\$0.50 \\
\midrule
\multicolumn{2}{l}{Profit Calculation:} & & & \\
\multicolumn{2}{l}{Cash:} & \$3.25 & & \\
\multicolumn{2}{l}{Futures:} & & \$3.75 - \$3.75 = \$0.00 & \\
\multicolumn{2}{l}{\textbf{Net per bushel:}} & \multicolumn{2}{l}{\$3.25 + \$0.00 = \textbf{\$3.25}} & \\
\bottomrule
\end{tabular}
\end{table}

\vspace{0.5em}

\textbf{Basis widening was a loss to the farmer}, even though general price levels unchanged.
\end{frame}

\begin{frame}{Case 4: Futures Unchanged, Basis Narrows to \$0.00}
\begin{table}
\centering
\footnotesize
\begin{tabular}{p{2cm}p{2.5cm}p{2cm}p{2.5cm}p{1.8cm}}
\toprule
\textbf{Date} & \textbf{Action} & \textbf{Cash Price} & \textbf{Dec Futures} & \textbf{Basis} \\
\midrule
May 1st & Sell Dec Futures & \$3.50 & \$3.75 & -\$0.25 \\
Nov 1st & Buy Dec Futures \& Sell Cash Corn & \$3.75 & \$3.75 & \$0.00 \\
\midrule
\multicolumn{2}{l}{Profit Calculation:} & & & \\
\multicolumn{2}{l}{Cash:} & \$3.75 & & \\
\multicolumn{2}{l}{Futures:} & & \$3.75 - \$3.75 = \$0.00 & \\
\multicolumn{2}{l}{\textbf{Net per bushel:}} & \multicolumn{2}{l}{\$3.75 + \$0.00 = \textbf{\$3.75}} & \\
\bottomrule
\end{tabular}
\end{table}

\vspace{0.5em}

\textbf{Narrowing of the basis was an increase in profit} to the farmer.
\end{frame}

\begin{frame}{Long the Basis}
When a farmer hedges with futures, they are ``\textbf{long in the basis}''.

\vspace{1em}

A futures hedge eliminates price uncertainty from general price levels:
\begin{itemize}
    \item If prices go up $\rightarrow$ increase in cash price, but loss on futures position
    \item If prices go down $\rightarrow$ receive less in cash, but gain on futures position
\end{itemize}

\vspace{1em}

\textbf{However:} The futures hedge does NOT protect against changes (good or bad) in the relative price between the local cash market and the futures price.
\end{frame}

%------------------------------------------------------------
% FLOUR MILL HEDGING
%------------------------------------------------------------

\begin{frame}{}
\centering
\Huge Flour Mill's Hedge
\end{frame}

\begin{frame}{The Flour Mill's Situation}
A flour mill:
\begin{itemize}
    \item Buys large quantities of grain for making flour
    \item Wants to process grain year round
    \item Needs to hedge price risk at multiple points in time
\end{itemize}

\vspace{1em}

\textbf{Example:} High commercial wheat flour mills can process upwards of \textbf{50,000 bushels} of wheat per month.

\vspace{0.5em}

With wheat futures at 5,000 bushels/contract, they need \textbf{10 contracts} to hedge one month's wheat buying.
\end{frame}

\begin{frame}{Hedging by Buying Ahead}
\begin{block}{Flour Mill Hedge Rule}
Remember: A futures hedge always involves making a trade in the futures contract that is the same as what you will do in the cash market.
\end{block}

\vspace{1em}

The flour mill \textbf{buys} grain, so their futures hedge should \textbf{buy} futures (``buying ahead'').
\end{frame}

\begin{frame}{Wheat Futures Contracts}
There are \textbf{five wheat futures contracts per year}:

\vspace{1em}

\begin{table}
\centering
\begin{tabular}{ccccc}
\toprule
March & May & July & September & December \\
\bottomrule
\end{tabular}
\end{table}

\vspace{1em}

To hedge, the mill must use the \textbf{nearest contract month}.

\vspace{0.5em}

Example: To hedge a February 1st purchase, use March futures (no February contract exists).
\end{frame}

\begin{frame}{Hedging Strategy Example}
\textbf{On December 1st:} Mill wants to lock in wheat purchase price for first six months of next year.

\vspace{1em}

\begin{itemize}
    \item January, February, March purchases $\rightarrow$ hedge with March futures
    \item 50,000 bu/month $\times$ 3 months = 150,000 bu = 30 contracts
    \item April, May purchases $\rightarrow$ hedge with May futures (20 contracts)
    \item June purchase $\rightarrow$ hedge with July futures (10 contracts)
\end{itemize}

\vspace{0.5em}

\textbf{Total:} Buy 60 contracts on Dec 1st to hedge Jan--Jun purchases.
\end{frame}

\begin{frame}{Lifting the Hedge}
Hedges should be \textbf{lifted simultaneously} with activity in the cash market.

\vspace{1em}

\textbf{Example:}
\begin{itemize}
    \item On January 1st: Mill purchases 50,000 bu in cash market
    \item Simultaneously: Sell 10 March contracts (lift the hedge)
    \item After January 1st: Still holding 20 long March contracts for Feb \& Mar purchases
\end{itemize}
\end{frame}

\begin{frame}{Flour Mill Hedge: Real Data 2016}
\begin{table}
\centering
\tiny
\begin{tabular}{p{1.2cm}p{2.8cm}p{2.5cm}p{2.8cm}p{0.8cm}p{0.8cm}p{0.9cm}}
\toprule
\textbf{Date} & \textbf{Action} & \textbf{Long Position} & \textbf{Net Price Paid} & \textbf{Spot} & \textbf{Mar} & \textbf{May} \\
\midrule
Dec 1, 15 & Buy 30 Mar, 20 May, 10 Jul & & & 469.75 & 470 & 476.5 \\
Jan 1, 16 & Buy 50k spot, Sell 10 Mar & 20 Mar, 20 May, 10 Jul & -479 + (479.25-470) = -469.75 & 479.00 & 479.25 & 485 \\
Feb 1 & Buy 50k spot, Sell 10 Mar & 10 Mar, 20 May, 10 Jul & -444.75 + (445-470) = -469.75 & 444.75 & 445 & 453.25 \\
Mar 1 & Buy 50k spot, Sell 10 Mar & 20 May, 10 Jul & -471.25 + (471.5-470) = -469.75 & 471.25 & 471.5 & 473.5 \\
Apr 1 & Buy 50k spot, Sell 10 May & 10 May, 10 Jul & -477.75 + (478-476.5) = -476.25 & 477.75 & NA & 478 \\
May 1 & Buy 50k spot, Sell 10 May & 10 Jul & -464.75 + (465-476.5) = -476.25 & 464.75 & NA & 465 \\
Jun 1 & Buy 50k spot, Sell 10 Jul & 0 & -431 + (431.25-483.25) = -483 & 431 & NA & NA \\
\bottomrule
\end{tabular}
\end{table}

\vspace{0.5em}
\tiny{Assumes basis is fixed at -0.25 cents under the nearby futures contract.}
\end{frame}

%------------------------------------------------------------
% SOYBEAN CRUSHER
%------------------------------------------------------------

\begin{frame}{}
\centering
\Huge Soybean Crusher
\end{frame}

\begin{frame}{Soybean Crusher's Hedge}
The soybean crusher:
\begin{itemize}
    \item \textbf{Buys} soybeans
    \item \textbf{Sells} soybean meal and oil
\end{itemize}

\vspace{1em}

Their futures hedge would involve:
\begin{itemize}
    \item \textbf{Buying} soybean futures (hedge the input)
    \item \textbf{Selling} meal and oil futures (hedge the outputs)
\end{itemize}

\vspace{1em}

``Crush'' hedges are more complicated because they involve buying one commodity, transforming it, and selling another commodity.
\end{frame}

%------------------------------------------------------------
% LINE OF CREDIT
%------------------------------------------------------------

\begin{frame}{}
\centering
\Huge Importance of Line of Credit
\end{frame}

\begin{frame}{Why a Line of Credit Matters}
Anyone hedging with futures must have a \textbf{line of credit} to meet margin calls when the market moves against their hedge.

\vspace{1em}

\textbf{Example:} Farmer sells futures as a hedge
\begin{itemize}
    \item If price starts moving up $\rightarrow$ cash grain worth more, but futures hedge is losing money
    \item As short hedge loses money $\rightarrow$ margin needs to be maintained
    \item If prices move significantly against hedge $\rightarrow$ additional money needed for margin
\end{itemize}
\end{frame}

\begin{frame}{The Role of the Lender}
\begin{itemize}
    \item Additional margin usually must come from a \textbf{lender}
    \item Lender knows it's a safe loan because prices are going up---the cash sale will cover any futures losses
\end{itemize}

\vspace{1em}

\textbf{Without a line of credit:}
\begin{itemize}
    \item Cannot maintain margin
    \item Short hedge position will be \textbf{forced to liquidate}
    \item Farmer will no longer be hedged
\end{itemize}
\end{frame}

%------------------------------------------------------------
% SUMMARY
%------------------------------------------------------------

\begin{frame}{Summary}
\begin{itemize}
    \item \textbf{Futures contracts:} Agreements to buy/sell at an agreed price; profits/losses calculated by price movement $\times$ contract size
    \item \textbf{Delivery:} Most contracts closed before expiration; actual delivery uses shipping certificates
    \item \textbf{Hedging rule:} Take the same action in futures as you will in the cash market
    \item \textbf{Farmer (seller):} Sells futures to hedge; is ``long the basis''
    \item \textbf{Flour mill (buyer):} Buys futures to hedge
    \item \textbf{Basis risk:} Hedging eliminates price level risk but not basis risk
    \item \textbf{Line of credit:} Essential to maintain margin when market moves against the hedge
\end{itemize}
\end{frame}

\end{document}
