\documentclass[aspectratio=169]{beamer}
\usetheme{Madrid}

\usepackage{graphicx}
\usepackage{booktabs}
\usepackage{hyperref}

% Purdue Colors
\definecolor{PurdueGold}{HTML}{CEB888}
\definecolor{PurdueBlack}{HTML}{000000}

% Apply Purdue color theme
\setbeamercolor{palette primary}{bg=PurdueBlack,fg=PurdueGold}
\setbeamercolor{palette secondary}{bg=PurdueGold,fg=PurdueBlack}
\setbeamercolor{palette tertiary}{bg=PurdueBlack,fg=PurdueGold}
\setbeamercolor{palette quaternary}{bg=PurdueGold,fg=PurdueBlack}
\setbeamercolor{structure}{fg=PurdueBlack}
\setbeamercolor{section in toc}{fg=PurdueBlack}
\setbeamercolor{title}{fg=PurdueGold,bg=PurdueBlack}
\setbeamercolor{frametitle}{fg=PurdueGold,bg=PurdueBlack}
\setbeamercolor{block title}{bg=PurdueBlack,fg=PurdueGold}
\setbeamercolor{block body}{bg=PurdueGold!20,fg=PurdueBlack}
\setbeamercolor{item}{fg=PurdueBlack}
\setbeamercolor{subitem}{fg=PurdueBlack}
\setbeamercolor{subsubitem}{fg=PurdueBlack}

\title{Grain and Oilseed Markets}
\subtitle{A Primer for Grain and Oilseed Price Analysis}
\author{Mindy Mallory}
\date{}

\begin{document}

\begin{frame}
\titlepage
\end{frame}

\begin{frame}{Highlights}
\begin{itemize}
    \item Biological processes of corn, soybeans, and wheat from planting, to growing, to harvesting, to storage
    \item How weather interacts with the biological processes to determine production
    \item Recent trends in production, exports, and prices
    \item Varieties of wheat and determinants of their differential prices
\end{itemize}
\end{frame}

\begin{frame}{Check Your Understanding}
\begin{itemize}
    \item When are corn and soybeans planted?
    \item When are corn and soybeans harvested?
    \item When are the different varieties of wheat planted and harvested?
    \item What is the main driver of differential prices in wheat varieties?
\end{itemize}
\end{frame}

\begin{frame}{Why Biology Matters for Prices}
Commodities are natural things subject to biological processes.

\vspace{1em}

You must understand the basic biological processes involved in a commodity's production in order to understand and anticipate what happens to its price.
\end{frame}

%------------------------------------------------------------
% CORN SECTION
%------------------------------------------------------------

\begin{frame}{}
\centering
\Huge Corn
\end{frame}

\begin{frame}{Corn: Production Timeline}
\begin{itemize}
    \item \textbf{Planted:} March to May
    \item \textbf{Pollination:} July
    \item \textbf{Harvested:} September to October
\end{itemize}

\vspace{1em}

Since pollination is key to production and yield, new crop futures prices tend to be highly variable in June and July as weather patterns mean the difference between a high yielding year and low yielding year.
\end{frame}

\begin{frame}{Corn: Weather Matters}
\textbf{Planting Weather}
\begin{itemize}
    \item Too wet $\rightarrow$ difficult to get acreage planted in a timely manner
    \item Corn planted too late may suffer a yield penalty
\end{itemize}

\vspace{1em}

\textbf{Harvest Weather}
\begin{itemize}
    \item Very wet conditions $\rightarrow$ difficult to get the crop out and dry before it is damaged
\end{itemize}
\end{frame}

\begin{frame}{Corn Acres Planted 2019}
\centering
\includegraphics[width=0.7\textwidth]{../../assets/map_CORN_AREA PLANTED.png}

\tiny{Source: Aaron Smith's Ag Data}
\end{frame}

\begin{frame}{Corn: Acreage Dynamics}
\begin{itemize}
    \item U.S. corn planted acres: just under to just over 90 million acres in recent years
    \item Corn and soybeans ``compete for acres'' in the Corn Belt
    \item Farmers shift acres based on relative new crop futures prices
    \item High corn acres $\rightarrow$ lower soybean acres (and vice versa)
\end{itemize}
\end{frame}

\begin{frame}{Corn: Yield Improvements}
\begin{itemize}
    \item Seed hybrids and genetic modification have led to dramatic yield increases over the last 100 years
    \item Corn planted acres have been relatively flat for a very long time
    \item But production has skyrocketed
\end{itemize}
\end{frame}

\begin{frame}{Corn: Acreage and Yields}
\centering
\includegraphics[width=0.85\textwidth]{../../assets/PrimerforGrain_CornAcandY.png}

\tiny{Data from USDA NASS}
\end{frame}

\begin{frame}{Corn: Production and Prices}
\centering
\includegraphics[width=0.85\textwidth]{../../assets/PrimerforGrain_CornProdand.png}

\vspace{0.5em}
Corn prices can be quite volatile, with prices and production highly inversely related.

\tiny{Data from USDA NASS}
\end{frame}

\begin{frame}{Corn: Uses}
\centering
\includegraphics[width=0.85\textwidth]{../../assets/PrimerforGrain_CornUse.png}

\vspace{0.5em}
Largest use categories: feed (livestock) and alcohol for fuel use (ethanol).

\tiny{Data from USDA ERS}
\end{frame}

%------------------------------------------------------------
% SOYBEANS SECTION
%------------------------------------------------------------

\begin{frame}{}
\centering
\Huge Soybeans
\end{frame}

\begin{frame}{Soybeans: Production Timeline}
\begin{itemize}
    \item \textbf{Planted:} April to June (later than corn)
    \item \textbf{Weather effects:} Similar to corn
    \item Soybean prices are highly dependent on what happens during summer months
\end{itemize}
\end{frame}

\begin{frame}{Soybean Acres Planted 2019}
\centering
\includegraphics[width=0.7\textwidth]{../../assets/map_SOYBEANS_AREA PLANTED.png}

\tiny{Source: Aaron Smith's Ag Data}
\end{frame}

\begin{frame}{Soybeans: Acreage and Yields}
\begin{itemize}
    \item Soybeans were not commonly planted in the U.S. until the mid 20th century
    \item Once introduced, acreage expanded rapidly
    \item Soybeans have also benefited from improved yields due to biotechnology
\end{itemize}
\end{frame}

\begin{frame}{Soybeans: Acreage and Yields}
\centering
\includegraphics[width=0.85\textwidth]{../../assets/PrimerforGrain_SoyAcandY.png}

\tiny{Data from USDA NASS}
\end{frame}

\begin{frame}{Soybeans: Production and Prices}
\centering
\includegraphics[width=0.85\textwidth]{../../assets/PrimerforGrain_SoyProdand.png}

\tiny{Data from USDA NASS}
\end{frame}

\begin{frame}{Soybeans: Uses}
\textbf{Domestic Consumption}
\begin{itemize}
    \item Almost exclusively processed into soybean meal and soybean oil (``crushing'')
    \item Soybean meal: high protein, used in livestock feed
    \item Soybean oil: bulk consumed as edible oil
\end{itemize}

\vspace{1em}

\textbf{Exports}
\begin{itemize}
    \item About half of U.S. soybeans are exported
    \item More than half of exports go to China
\end{itemize}
\end{frame}

\begin{frame}{Soybeans: Uses}
\centering
\includegraphics[width=0.85\textwidth]{../../assets/PrimerforGrain_SoyUse.png}

\tiny{Data from USDA ERS}
\end{frame}

%------------------------------------------------------------
% WHEAT SECTION
%------------------------------------------------------------

\begin{frame}{}
\centering
\Huge Wheat
\end{frame}

\begin{frame}{Wheat: Three Main Types in the U.S.}
\begin{enumerate}
    \item \textbf{Hard Red Winter Wheat} (HRW / KC Wheat)
    \item \textbf{Hard Red Spring Wheat} (HRS / Minneapolis Wheat)
    \item \textbf{Soft Red Winter Wheat} (SRW / Chicago Wheat)
\end{enumerate}

\vspace{1em}

Each type:
\begin{itemize}
    \item Has its own futures contract
    \item Grown in distinct regions
    \item Has different end uses
    \item Varies in protein content
\end{itemize}
\end{frame}

\begin{frame}{Wheat Growing Areas}
\centering
\includegraphics[width=0.65\textwidth]{../../images/Wheat-Growing-Areas.png}

\vspace{0.5em}
Blue = HRW \quad Gold = HRS \quad Red = SRW

\tiny{Source: USDA-ERS}
\end{frame}

\begin{frame}{Hard Red Winter Wheat (KC Wheat)}
\begin{columns}
\begin{column}{0.6\textwidth}
\textbf{Production}
\begin{itemize}
    \item Planted in fall, dormant/slow growth in winter
    \item Grows as temperatures rise in spring
    \item Heads form in April--May
    \item Harvested when plants die
\end{itemize}

\vspace{0.5em}

\textbf{Use:} Primarily bread flour

\vspace{0.5em}

Called ``Kansas City Wheat'' because KCBOT had the HRW futures contract (now CME Group)
\end{column}
\begin{column}{0.4\textwidth}
\centering
\includegraphics[width=0.8\textwidth]{../../images/HRW-Wheat.jpg}

\tiny{Source: USDA-GIPSA}
\end{column}
\end{columns}
\end{frame}

\begin{frame}{Hard Red Spring Wheat (Minneapolis Wheat)}
\begin{columns}
\begin{column}{0.6\textwidth}
\textbf{Production}
\begin{itemize}
    \item Planted in spring (April--May)
    \item Harvested in fall (September)
\end{itemize}

\vspace{0.5em}

\textbf{Characteristics}
\begin{itemize}
    \item Highest protein content (13--16\%)
    \item High gluten content
\end{itemize}

\vspace{0.5em}

\textbf{Use:} Bread baking; blending with lower protein wheat

\vspace{0.5em}

Called ``Minneapolis Wheat'' because Minneapolis Grain Exchange offers HRS futures
\end{column}
\begin{column}{0.4\textwidth}
\centering
\includegraphics[width=0.8\textwidth]{../../images/HRS-Wheat.jpg}

\tiny{Source: USDA-GIPSA}
\end{column}
\end{columns}
\end{frame}

\begin{frame}{Soft Red Winter Wheat (Chicago Wheat)}
\begin{columns}
\begin{column}{0.6\textwidth}
\textbf{Production}
\begin{itemize}
    \item Planted in fall
    \item Harvested in late spring (like HRW)
\end{itemize}

\vspace{0.5em}

\textbf{Characteristics}
\begin{itemize}
    \item Lower in protein
\end{itemize}

\vspace{0.5em}

\textbf{Use:} Cakes, cookies, crackers (high gluten not required)

\vspace{0.5em}

Called ``Chicago Wheat'' because CBOT offers SRW futures
\end{column}
\begin{column}{0.4\textwidth}
\centering
\includegraphics[width=0.8\textwidth]{../../images/SRW-Wheat.jpg}

\tiny{Source: USDA-GIPSA}
\end{column}
\end{columns}
\end{frame}

\begin{frame}{Protein Premiums and Wheat Spreads}
Flour millers rarely use just one kind of wheat---they blend different types to make flours of varying protein and gluten levels.

\vspace{1em}

\textbf{Key dynamic:}
\begin{itemize}
    \item High yields $\rightarrow$ lower wheat protein
    \item Good winter wheat yields $\rightarrow$ plenty of wheat but not enough protein
    \item This causes spring wheat prices to rise against winter wheat prices
    \item If winter wheat crops are smaller $\rightarrow$ protein content is higher $\rightarrow$ spring wheat may not enjoy a large premium
\end{itemize}
\end{frame}

\begin{frame}{Protein Premiums and Wheat Spreads}
Because of the protein/yield dynamic:

\vspace{1em}

The relative prices of Minneapolis, Kansas City, and Chicago wheat futures are closely followed by stakeholders in any of the wheat markets.
\end{frame}

\begin{frame}{Wheat: Acreage and Yields}
\centering
\includegraphics[width=0.85\textwidth]{../../assets/PrimerforGrain_WheatAcandY.png}

\tiny{Data from USDA NASS}
\end{frame}

\begin{frame}{Wheat: Production and Prices}
\centering
\includegraphics[width=0.85\textwidth]{../../assets/PrimerforGrain_WheatProdand.png}

\tiny{Data from USDA NASS}
\end{frame}

%------------------------------------------------------------
% SUMMARY
%------------------------------------------------------------

\begin{frame}{Summary}
\begin{itemize}
    \item \textbf{Corn:} Planted Mar--May, harvested Sep--Oct. Pollination (July) is critical. Used for feed and ethanol.
    \item \textbf{Soybeans:} Planted Apr--Jun. Crushed into meal and oil. About half exported, mostly to China.
    \item \textbf{Wheat:} Three main types with different protein levels and uses. Protein premiums drive relative prices.
\end{itemize}

\vspace{1em}

Understanding the biological processes and timing is essential for understanding price movements.
\end{frame}

\end{document}
