\documentclass[aspectratio=169]{beamer}
\usetheme{Madrid}

\usepackage{graphicx}
\usepackage{booktabs}
\usepackage{hyperref}

% Purdue Colors
\definecolor{PurdueGold}{HTML}{CEB888}
\definecolor{PurdueBlack}{HTML}{000000}

% Apply Purdue color theme
\setbeamercolor{palette primary}{bg=PurdueBlack,fg=PurdueGold}
\setbeamercolor{palette secondary}{bg=PurdueGold,fg=PurdueBlack}
\setbeamercolor{palette tertiary}{bg=PurdueBlack,fg=PurdueGold}
\setbeamercolor{palette quaternary}{bg=PurdueGold,fg=PurdueBlack}
\setbeamercolor{structure}{fg=PurdueBlack}
\setbeamercolor{section in toc}{fg=PurdueBlack}
\setbeamercolor{title}{fg=PurdueGold,bg=PurdueBlack}
\setbeamercolor{frametitle}{fg=PurdueGold,bg=PurdueBlack}
\setbeamercolor{block title}{bg=PurdueBlack,fg=PurdueGold}
\setbeamercolor{block body}{bg=PurdueGold!20,fg=PurdueBlack}
\setbeamercolor{item}{fg=PurdueBlack}
\setbeamercolor{subitem}{fg=PurdueBlack}
\setbeamercolor{subsubitem}{fg=PurdueBlack}

\title{How to Find Information}
\subtitle{Data Sources and Market Commentary}
\author{Mindy Mallory}
\date{}

\begin{document}

\begin{frame}
\titlepage
\end{frame}

\begin{frame}{Chapter Overview}
\begin{itemize}
    \item Real-time and historical data sources
    \item Professional analysts and market commentary
    \item Futures markets as source of real-time price information
    \item USDA reports and databases
\end{itemize}

\vspace{1em}

The more you follow commodity markets, the more you learn. Absorb the insights of professionals who live and breathe commodity markets every day.
\end{frame}

%------------------------------------------------------------
% MARKET COMMENTARY
%------------------------------------------------------------

\begin{frame}{}
\centering
\Huge Market Commentary
\end{frame}

\begin{frame}{Market Commentary Sources}
The best way to learn commodity price analysis is to listen to the professionals.

\vspace{1em}

\begin{itemize}
    \item Land grant universities in major commodity producing states
    \item Public radio in major commodity producing areas
\end{itemize}
\end{frame}

\begin{frame}{Key Resources for Market Commentary}
\begin{table}
\small
\begin{tabular}{ll}
\toprule
\textbf{Outlet} & \textbf{Description} \\
\midrule
Farmdoc Daily & U of I Dept.\ of ACE extension \\
WILL Agriculture & WILL and U of I Extension \\
Market to Market & Iowa Public Television \\
Center for Commercial Ag & Purdue University Ag Econ \\
\bottomrule
\end{tabular}
\end{table}
\end{frame}

%------------------------------------------------------------
% FUTURES PRICE QUOTES
%------------------------------------------------------------

\begin{frame}{}
\centering
\Huge Futures Price Quotes
\end{frame}

\begin{frame}{Futures vs.\ Forward Contracts}
\textbf{Futures contracts:}
\begin{itemize}
    \item Quantity and quality are standardized
    \item Traded on an exchange
    \item Exchange becomes seller to every buyer and buyer to every seller
    \item Easy to get into and out of positions
    \item Take offsetting position at prevailing price
\end{itemize}

\vspace{1em}

\textbf{Forward contracts:}
\begin{itemize}
    \item Specific counter-parties (buyers and sellers)
    \item Customized terms
\end{itemize}
\end{frame}

\begin{frame}{CME Group}
For U.S.\ agricultural commodities, the CME Group is the most important futures exchange for price discovery.

\vspace{1em}

\textbf{Contracts traded:}
\begin{itemize}
    \item Grains/Oilseeds: corn, soybeans, soybean oil, soybean meal, SRW wheat, HRW wheat
    \item Livestock: live cattle, lean hogs, feeder cattle
    \item Softs: cocoa, coffee, sugar
    \item Energy: crude oil, natural gas, ethanol
\end{itemize}
\end{frame}

\begin{frame}{World Prices}
Chicago Board of Trade (owned by CME Group) futures prices of corn and soybeans are considered the ``World Price.''

\vspace{1em}

All over the globe, prices for these commodities are set based on what the price of corn and soybeans are trading on the CME Group exchanges.
\end{frame}

\begin{frame}{Where to Get Price Quotes}
\textbf{CME Group website:}
\begin{itemize}
    \item Real-time data (10-min delay)
    \item Most recent quotes
    \item Some charting capability
\end{itemize}

\vspace{1em}

\textbf{Third party vendors (more convenient):}
\begin{itemize}
    \item Barchart
    \item TradingView
    \item More flexible interface
    \item Ability to download recent price history
\end{itemize}
\end{frame}

%------------------------------------------------------------
% FUTURES SYMBOLS
%------------------------------------------------------------

\begin{frame}{}
\centering
\Huge Futures Symbols
\end{frame}

\begin{frame}{Futures Ticker Convention}
Contracts for several different expiry dates trade at the same time.

\vspace{1em}

\textbf{General convention:}
\begin{center}
\Large [Commodity Symbol] [Month Code] [Year]
\end{center}

\vspace{1em}

\textbf{Example:} CZ21
\begin{itemize}
    \item C = Corn
    \item Z = December
    \item 21 = 2021
\end{itemize}
\end{frame}

\begin{frame}{Month Codes}
\begin{table}
\begin{tabular}{llll}
\toprule
\textbf{Month} & \textbf{Code} & \textbf{Month} & \textbf{Code} \\
\midrule
January & F & July & N \\
February & G & August & Q \\
March & H & September & U \\
April & J & October & V \\
May & K & November & X \\
June & M & December & Z \\
\bottomrule
\end{tabular}
\end{table}
\end{frame}

\begin{frame}{Common Contract Symbols (Barchart Style)}
\begin{itemize}
    \item \textbf{Corn:} ZC --- Months: H, K, N, U, Z
    \item \textbf{Soybeans:} ZS --- Months: F, H, K, N, Q, U, X
    \item \textbf{SRW Wheat:} ZW --- Months: H, K, N, U, Z
    \item \textbf{HRW Wheat:} KE --- Months: H, K, N, U, Z
    \item \textbf{HRS Wheat:} MW --- Months: H, K, N, U, Z
    \item \textbf{Live Cattle:} LE --- Months: G, J, M, Q, V, Z
    \item \textbf{Lean Hogs:} HE --- Months: G, J, K, M, N, Q, V, Z
\end{itemize}
\end{frame}

\begin{frame}{Commodity Spreads}
\begin{table}
\small
\begin{tabular}{ll}
\toprule
\textbf{Spread} & \textbf{Source} \\
\midrule
Soybean Crush & Barchart Spread Chart \\
Cattle Crush & ISU Spread Calculation \\
Corn Crush (Ethanol) & ISU Ethanol Grind Margin \\
\bottomrule
\end{tabular}
\end{table}

\vspace{1em}

We will cover these spreads in detail later in the course.
\end{frame}

%------------------------------------------------------------
% USDA REPORTS
%------------------------------------------------------------

\begin{frame}{}
\centering
\Huge USDA Reports
\end{frame}

\begin{frame}{USDA's Role}
The USDA has a long history of:
\begin{itemize}
    \item Extensively surveying market conditions
    \item Reporting to the public
    \item Maintaining accessible databases
\end{itemize}

\vspace{1em}

Because of their efforts to provide consistent and accurate estimates of key variables, market participants follow USDA reports very closely.

\vspace{1em}

Their impacts on prices can be seen immediately---sometimes causing rapid, if not instantaneous price moves.
\end{frame}

\begin{frame}{Key Variables in USDA Reports}
\begin{itemize}
    \item Acres planted
    \item Yield
    \item Production
    \item Stocks
    \item Consumption
    \item Exports
\end{itemize}
\end{frame}

\begin{frame}{Prospective Plantings Report}
\begin{itemize}
    \item Estimate of grower intentions for planted acres
    \item Based on surveys during first two weeks of March
    \item Released on the last day of March every year
\end{itemize}

\vspace{1em}

Why do prospective plantings change year-to-year?
\begin{itemize}
    \item Relative prices of crops that compete for acres
    \item Growers look at harvest-time futures prices to estimate expected profit
    \item ``Corn is bidding for acres''
\end{itemize}
\end{frame}

\begin{frame}{Acreage Report}
\begin{itemize}
    \item Estimate of acres actually planted (vs.\ intentions)
    \item Based on surveys during first two weeks of June
    \item Released on the last day of June
\end{itemize}

\vspace{1em}

Why might Acreage differ from Prospective Plantings?
\begin{itemize}
    \item Weather (wet spring makes planting difficult)
    \item Crops competing for acres planted at different times
    \item Relative prices can change between March and June
\end{itemize}
\end{frame}

\begin{frame}{Grain Stocks Report}
\begin{itemize}
    \item Issued quarterly
    \item How much of selected commodities are in storage in the U.S., by state
\end{itemize}

\vspace{1em}

\textbf{Pertinent to:}
\begin{itemize}
    \item Price level (tight stocks $\rightarrow$ high prices)
    \item Calendar spreads between futures maturities
\end{itemize}

\vspace{1em}

When stocks are tight, nearby futures should exceed distant futures (incentivizes bringing stocks to market).
\end{frame}

\begin{frame}{WASDE Report}
\textbf{World Agricultural Supply and Demand Estimates}

\vspace{1em}

\begin{itemize}
    \item Released monthly
    \item USDA's forecasts for U.S.\ and world supply and use balance sheets
    \item Covers grains, soybeans and products, cotton, sugar, livestock
    \item Among the most important and eagerly anticipated reports
    \item Released at 12pm EST
    \item Commonly results in limit price moves in futures markets
\end{itemize}
\end{frame}

\begin{frame}{WASDE: Lock-up Conditions}
\begin{quote}
\small
To assure the highly market-sensitive information is released simultaneously to all end-users, and not prematurely to any one, the WASDE report is prepared under tight security in a specially designed area of USDA's South Building. The morning of release, doors in the ``lockup'' area are secured, window shades are sealed, and telephone and Internet communications are blocked.
\end{quote}

\vspace{0.5em}

\hfill --- USDA Office of the Chief Economist
\end{frame}

\begin{frame}{Trading Places (1983)}
The prospect that USDA reports could be leaked is the inspiration behind the final scene in \textit{Trading Places}.

\vspace{1em}

\begin{itemize}
    \item Eddie Murphy and Dan Aykroyd learn antagonists obtained advance access to a ``crop report'' for orange juice futures
    \item They intercept and feed false information
    \item Antagonists buy futures expecting price to rise
    \item Report comes out, price crashes
    \item Murphy and Aykroyd profit, retire to tropical paradise
\end{itemize}
\end{frame}

\begin{frame}{Other Important USDA Reports}
\textbf{Crop Production}
\begin{itemize}
    \item Estimates of yield, acres harvested, total production
    \item Released in tandem with WASDE
\end{itemize}

\vspace{1em}

\textbf{Crop Progress and Condition}
\begin{itemize}
    \item Issued every Monday during planting, growing, and harvest season
    \item NASS surveys approximately 4,000 individuals
    \item Critical information about status of the crop
\end{itemize}

\vspace{1em}

\textbf{Agricultural Marketing Service (AMS)}
\begin{itemize}
    \item Numerous daily and weekly reports
    \item Regional prices and exports
\end{itemize}
\end{frame}

%------------------------------------------------------------
% USDA DATA SOURCES
%------------------------------------------------------------

\begin{frame}{}
\centering
\Huge USDA Data Sources
\end{frame}

\begin{frame}{USDA Agencies with Databases}
\textbf{NASS} --- National Agricultural Statistics Service
\begin{itemize}
    \item Acres planted, production, yield, price received
\end{itemize}

\vspace{0.5em}

\textbf{FAS} --- Foreign Agricultural Service
\begin{itemize}
    \item Production Supply and Distribution (PSD) report
    \item Much of WASDE archived here
\end{itemize}

\vspace{0.5em}

\textbf{ERS} --- Economic Research Service
\begin{itemize}
    \item Cost and returns data
\end{itemize}

\vspace{0.5em}

\textbf{AMS} --- Agricultural Marketing Service
\begin{itemize}
    \item Local prices
\end{itemize}
\end{frame}

\begin{frame}{Summary of USDA Reports}
\begin{table}
\footnotesize
\begin{tabular}{ll}
\toprule
\textbf{Report} & \textbf{Content} \\
\midrule
Prospective Plantings & Farmers' planting intentions (March) \\
Acreage & Planted acres (June) \\
WASDE & World supply and demand estimates (Monthly) \\
Crop Production & Acres and yield (with WASDE) \\
Crop Progress & Weekly crop status (Mondays) \\
Grain Stocks & Quarterly storage levels \\
\bottomrule
\end{tabular}
\end{table}
\end{frame}

%------------------------------------------------------------
% SUMMARY
%------------------------------------------------------------

\begin{frame}{Summary}
\begin{itemize}
    \item Follow professional market commentary (Farmdoc Daily, WILL, Market to Market)
    \item CME Group is the primary futures exchange for U.S.\ ag commodities
    \item Learn futures ticker conventions: [Symbol][Month][Year]
    \item USDA reports move markets---follow them closely
    \item Key reports: Prospective Plantings, Acreage, WASDE, Crop Production, Crop Progress, Grain Stocks
    \item USDA databases (NASS, FAS, ERS, AMS) provide historical data for analysis
\end{itemize}
\end{frame}

\end{document}
