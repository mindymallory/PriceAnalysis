\documentclass[aspectratio=169,10pt]{beamer}

%% =============================================================================
%% PURDUE UNIVERSITY BEAMER THEME
%% Colors: Black and Old Gold
%% =============================================================================

%% --- Packages ---
\usepackage[utf8]{inputenc}
\usepackage[T1]{fontenc}
\usepackage{amsmath,amssymb,amsthm}
\usepackage{graphicx}
\usepackage{booktabs}
\usepackage{listings}
\usepackage{xcolor}
\usepackage{hyperref}
\usepackage{tikz}

%% --- Purdue Colors ---
\definecolor{PurdueGold}{RGB}{207,185,145}      % #CFB991 - Old Gold (primary)
\definecolor{PurdueGoldDark}{RGB}{177,129,11}   % #B1810B - Darker Gold (headers, accents)
\definecolor{PurdueBlack}{RGB}{0,0,0}           % #000000
\definecolor{PurdueGray}{RGB}{85,87,89}         % #555759 - Cool Gray
\definecolor{PurdueGoldLight}{RGB}{232,219,192} % Lighter gold for backgrounds

%% --- Theme Settings ---
\usetheme{Madrid}
\usecolortheme{default}

% Set colors throughout
\setbeamercolor{structure}{fg=PurdueGoldDark}
\setbeamercolor{palette primary}{bg=PurdueBlack,fg=PurdueGold}
\setbeamercolor{palette secondary}{bg=PurdueGoldDark,fg=white}
\setbeamercolor{palette tertiary}{bg=PurdueGold,fg=PurdueBlack}
\setbeamercolor{palette quaternary}{bg=PurdueBlack,fg=PurdueGold}

% Title page
\setbeamercolor{title}{fg=PurdueGold,bg=PurdueBlack}
\setbeamercolor{subtitle}{fg=PurdueGoldLight}
\setbeamercolor{author}{fg=white}
\setbeamercolor{institute}{fg=PurdueGold}
\setbeamercolor{date}{fg=PurdueGold}

% Frame titles
\setbeamercolor{frametitle}{fg=PurdueGold,bg=PurdueBlack}
\setbeamercolor{framesubtitle}{fg=PurdueGoldDark}

% Footline
\setbeamercolor{footline}{bg=PurdueBlack,fg=PurdueGold}

% Blocks
\setbeamercolor{block title}{bg=PurdueGoldDark,fg=white}
\setbeamercolor{block body}{bg=PurdueGoldLight,fg=PurdueBlack}

\setbeamercolor{block title example}{bg=PurdueGray,fg=white}
\setbeamercolor{block body example}{bg=gray!10,fg=PurdueBlack}

\setbeamercolor{block title alerted}{bg=PurdueBlack,fg=PurdueGold}
\setbeamercolor{block body alerted}{bg=gray!10,fg=PurdueBlack}

% Items
\setbeamercolor{item}{fg=PurdueGoldDark}
\setbeamercolor{subitem}{fg=PurdueGray}
\setbeamercolor{itemize item}{fg=PurdueGoldDark}
\setbeamercolor{itemize subitem}{fg=PurdueGray}
\setbeamercolor{enumerate item}{fg=PurdueGoldDark}

% Navigation symbols off
\setbeamertemplate{navigation symbols}{}

% Itemize bullets
\setbeamertemplate{itemize item}{\color{PurdueGoldDark}$\blacktriangleright$}
\setbeamertemplate{itemize subitem}{\color{PurdueGray}$\triangleright$}

%% --- Code Listings ---
\lstset{
  basicstyle=\ttfamily\small,
  keywordstyle=\color{PurdueGoldDark}\bfseries,
  commentstyle=\color{PurdueGray}\itshape,
  stringstyle=\color{PurdueBlack},
  numbers=left,
  numberstyle=\tiny\color{PurdueGray},
  numbersep=5pt,
  frame=single,
  framerule=0.5pt,
  rulecolor=\color{PurdueGold},
  backgroundcolor=\color{PurdueGoldLight!30},
  breaklines=true,
  showstringspaces=false,
  tabsize=2
}

%% --- Theorem Environments ---
\theoremstyle{definition}
\newtheorem{defn}{Definition}
\newtheorem{prop}{Proposition}
\newtheorem{thm}{Theorem}
\newtheorem{lem}{Lemma}
\newtheorem{cor}{Corollary}
\newtheorem{rem}{Remark}

%% --- Custom Commands ---
\newcommand{\highlight}[1]{\textcolor{PurdueGoldDark}{\textbf{#1}}}
\newcommand{\E}{\mathbb{E}}
\newcommand{\Var}{\text{Var}}
\newcommand{\Cov}{\text{Cov}}

%% --- Title Page Setup ---
\title[Marketing, Hedging \& Crop Insurance]{Marketing, Hedging, and Crop Insurance}
\subtitle{Integrating Risk Management Tools}
\author{Mindy Mallory}
\institute{Department of Agricultural Economics \\ Purdue University}
\date{\today}

%% --- Custom Title Page ---
\setbeamertemplate{title page}{
  \begin{tikzpicture}[remember picture,overlay]
    \fill[PurdueBlack] (current page.north west) rectangle (current page.south east);

    % Gold accent bar
    \fill[PurdueGold] ([yshift=-2cm]current page.north west) rectangle ([yshift=-2.3cm]current page.north east);

    % Title
    \node[anchor=west,text=PurdueGold,font=\Huge\bfseries,text width=14cm] at ([xshift=1cm,yshift=-4cm]current page.north west) {\inserttitle};

    % Subtitle
    \node[anchor=west,text=PurdueGoldLight,font=\large] at ([xshift=1cm,yshift=-5.5cm]current page.north west) {\insertsubtitle};

    % Author and Institute
    \node[anchor=west,text=white,font=\normalsize] at ([xshift=1cm,yshift=-7cm]current page.north west) {\insertauthor};
    \node[anchor=west,text=PurdueGold,font=\small] at ([xshift=1cm,yshift=-7.8cm]current page.north west) {\insertinstitute};

    % Date
    \node[anchor=west,text=PurdueGray,font=\small] at ([xshift=1cm,yshift=-9cm]current page.north west) {\insertdate};

    % Bottom gold bar
    \fill[PurdueGold] ([yshift=0.5cm]current page.south west) rectangle ([yshift=0.2cm]current page.south east);
  \end{tikzpicture}
}

%% --- Section Title Frames ---
\AtBeginSection[]{
  \begin{frame}
    \begin{tikzpicture}[remember picture,overlay]
      \fill[PurdueBlack] (current page.north west) rectangle (current page.south east);
      \node[text=PurdueGold,font=\Huge\bfseries,text width=12cm,align=center] at (current page.center) {\insertsectionhead};
      \fill[PurdueGold] ([yshift=-1cm]current page.center) +(-3,0) rectangle +(3,-0.1);
    \end{tikzpicture}
  \end{frame}
}

%% =============================================================================
%% DOCUMENT BEGINS
%% =============================================================================
\begin{document}

% Title frame
\begin{frame}[plain]
  \titlepage
\end{frame}

% Outline
\begin{frame}{Outline}
  \tableofcontents
\end{frame}

%% =============================================================================
%% HIGHLIGHTS
%% =============================================================================

\begin{frame}{Chapter Highlights}
  \begin{block}{Key Takeaways}
    \begin{itemize}
      \item Purchasing crop insurance provides protection against \highlight{catastrophic loss} in the event of crop failure, and/or significant price decline between planting and harvest
      \item How much protection crop insurance alone provides varies based on \highlight{pre-harvest price movements}
      \item Crop insurance alone does not provide an optimal marketing plan for some scenarios of price movement planting-harvest
    \end{itemize}
  \end{block}
\end{frame}

%% =============================================================================
%% SECTION 1: HISTORY OF CROP INSURANCE
%% =============================================================================

\section{Brief History of Crop Insurance}

\begin{frame}{Federal Farm Support Programs: Historical Context}
  \begin{itemize}
    \item Currently, crop insurance dominates the landscape of federally sponsored farm programs for risk management
    \item History of market interventions and price supports in the United States is long and varied
    \item Understanding the economic and political motivations for the transition is important
  \end{itemize}

  \vspace{1em}
  \begin{block}{Transition From}
    \begin{itemize}
      \item Price supports
      \item Ad hoc disaster payments
    \end{itemize}
  \end{block}

  \begin{block}{Transition To}
    \begin{itemize}
      \item Federally subsidized crop insurance
    \end{itemize}
  \end{block}
\end{frame}

\begin{frame}{Why Move Away from Price Supports?}
  \begin{alertblock}{Problems with Price Supports}
    \begin{itemize}
      \item Distort markets by creating \highlight{inefficiency in production decisions}
      \item Generate ill will with trading partners
      \item Subsidize too much production
      \item Lower world prices to the detriment of farmers in other countries
    \end{itemize}
  \end{alertblock}

  \vspace{1em}
  \begin{alertblock}{Problems with Ad Hoc Disaster Payments}
    \begin{itemize}
      \item Not predictable in \highlight{timing} of when needed
      \item Not predictable in \highlight{size} of the need
    \end{itemize}
  \end{alertblock}

  \vspace{0.5em}
  \textit{Both created an appetite among lawmakers for a program whose cost was more predictable.}
\end{frame}

\begin{frame}{Why Crop Insurance Requires Federal Involvement}
  \begin{block}{Challenge 1: Private Market Limitations}
    Insurance markets can exist in the private sector when:
    \begin{itemize}
      \item Insured losses are relatively small (compared to premium collected)
      \item Losses are predictable from year-to-year
    \end{itemize}
  \end{block}

  \vspace{0.5em}
  \begin{alertblock}{Crop Insurance is Neither!}
    \begin{itemize}
      \item Losses are typically \highlight{highly correlated}
      \item When there is a crop failure, most farmers in the affected region experience a loss
      \item Hard for private insurance companies to ensure liquidity for widespread failures
      \item Without a federal backstop, private market cannot offer crop insurance
    \end{itemize}
  \end{alertblock}
\end{frame}

\begin{frame}{Early Crop Insurance Programs: 1930s--1980s}
  \begin{block}{Challenge 2: Insufficient Farmer Participation}
    \begin{itemize}
      \item Early federal crop insurance programs did not attract sufficient usage
      \item This did not prevent passage of additional ad hoc disaster payments
    \end{itemize}
  \end{block}

  \vspace{1em}
  \begin{alertblock}{The Problem}
    If taxpayers have to pay for:
    \begin{enumerate}
      \item Administration of the crop insurance program, \textbf{AND}
      \item Ad hoc disaster payments from time-to-time
    \end{enumerate}

    \vspace{0.5em}
    Then the program is:
    \begin{itemize}
      \item Not meeting its purpose
      \item Wasting taxpayer money by paying for two solutions
    \end{itemize}
  \end{alertblock}
\end{frame}

\begin{frame}{Key Legislative Reforms}
  \begin{block}{Federal Crop Insurance Act of 1980}
    Made purchase of crop insurance coverage \highlight{mandatory} for farmers to receive other program payments
  \end{block}

  \vspace{0.5em}
  \begin{block}{Crop Insurance Reform Act of 1994}
    \begin{itemize}
      \item Mandatory coverage requirement was \highlight{removed}
      \item Replaced with \highlight{premium subsidies}
    \end{itemize}
  \end{block}

  \vspace{0.5em}
  \begin{exampleblock}{Result}
    Congress increased subsidy rates to levels where farmers voluntarily purchase enough coverage to reduce the need for frequent disaster payments
  \end{exampleblock}

  \vspace{0.5em}
  \begin{block}{Current Status}
    Federal subsidies for crop insurance average \highlight{62\%} of the premium cost
  \end{block}
\end{frame}

%% =============================================================================
%% SECTION 2: CROP INSURANCE OFFERINGS
%% =============================================================================

\section{Overview of Typical Crop Insurance Offerings}

\begin{frame}{Yield Protection and Area Yield Protection}
  \begin{block}{Yield Protection}
    \begin{itemize}
      \item Guarantees against a loss of production less than a percent of \highlight{Actual Production History (APH)}
      \item Indemnities paid when production falls below the APH
      \item Payments based on that production year's \highlight{projected price}
        \begin{itemize}
          \item For corn and soybeans: average of new crop futures settlement prices during February
        \end{itemize}
    \end{itemize}
  \end{block}

  \vspace{1em}
  \begin{block}{Area Yield Protection}
    \begin{itemize}
      \item Similar to Yield Protection
      \item Yield guarantee based on \highlight{county yields}, not the individual producer's APH
    \end{itemize}
  \end{block}
\end{frame}

\begin{frame}{Revenue Protection and Area Revenue Protection}
  \begin{block}{Revenue Protection}
    \begin{itemize}
      \item Guarantees against loss of revenue below:
      \[
        \text{APH} \times \max(\text{Projected Price}, \text{Harvest Price})
      \]
      \item \highlight{Projected Price}: New crop futures price in February
      \item \highlight{Harvest Price}: New crop futures price at harvest
      \item Gained popularity because farmers are concerned with \highlight{revenue loss}, not just yield loss
    \end{itemize}
  \end{block}

  \vspace{1em}
  \begin{block}{Area Revenue Protection}
    \begin{itemize}
      \item Similar to Revenue Protection
      \item Revenue guarantee calculated against the \highlight{county APH} rather than producer's APH
    \end{itemize}
  \end{block}
\end{frame}

\begin{frame}{Revenue Protection with Harvest Price Exclusion}
  \begin{block}{Key Difference from Standard Revenue Protection}
    \begin{itemize}
      \item Same as Revenue Protection, \textbf{except}:
      \item Policy does \highlight{NOT} guarantee based on the Harvest Price
      \item Revenue is calculated based on:
      \[
        \text{Projected Price} \times \text{APH}
      \]
    \end{itemize}
  \end{block}

  \vspace{1em}
  \begin{alertblock}{Implication}
    Provides less upside protection if harvest prices rise above projected prices
  \end{alertblock}
\end{frame}

\begin{frame}{Summary: Crop Insurance Types}
  \begin{center}
    \begin{tabular}{lcc}
      \toprule
      \textbf{Insurance Type} & \textbf{Basis} & \textbf{Price Guarantee} \\
      \midrule
      Yield Protection & Individual APH & Projected Price \\
      Area Yield Protection & County APH & Projected Price \\
      Revenue Protection & Individual APH & Max(Projected, Harvest) \\
      Area Revenue Protection & County APH & Max(Projected, Harvest) \\
      RP w/ Harvest Exclusion & Individual APH & Projected Price Only \\
      \bottomrule
    \end{tabular}
  \end{center}
\end{frame}

%% =============================================================================
%% SECTION 3: MARKETING STRATEGIES WITH CROP INSURANCE
%% =============================================================================

\section{Pre-Harvest Marketing Strategies with Crop Insurance}

\begin{frame}{Marketing Insured Bushels}
  \begin{block}{Current State of U.S. Crop Insurance}
    Roughly \highlight{90\%} of U.S. cropland is insured in the Federal Crop Insurance program
  \end{block}

  \vspace{1em}
  \begin{alertblock}{Critical Warning: Marketing Uninsured Bushels}
    Marketing uninsured bushels is very risky in the event of a crop failure!

    \vspace{0.5em}
    \textbf{Example:}
    \begin{itemize}
      \item Forward contract to deliver 1,000 bushels at \$5.00/bu
      \item Yield losses result in only 500 bushels produced
      \item Cannot meet delivery obligation
      \item Must buy out the contract at potentially \highlight{high spot prices}
    \end{itemize}
  \end{alertblock}
\end{frame}

\begin{frame}{Why Revenue Protection Enables Marketing}
  \begin{block}{Revenue Protection Guarantee}
    With Revenue Protection, you are guaranteed:
    \[
      \text{Coverage \%} \times \text{APH} \times \max(\text{Base Price}, \text{Harvest Price})
    \]
  \end{block}

  \vspace{1em}
  \begin{exampleblock}{Benefit for Marketing}
    This revenue guarantee ensures you can at least \highlight{buy out your forward contract} in the event of a crop failure
  \end{exampleblock}

  \vspace{0.5em}
  \textit{Revenue Protection provides a floor that supports pre-harvest marketing decisions}
\end{frame}

\begin{frame}{When Prices Fall After Base Price is Set}
  \begin{block}{The Hedging Perspective}
    From a hedging perspective, the marketing decision is \highlight{trivial} if prices fall after the base price is set:

    \begin{itemize}
      \item You already have revenue protected at: \\
        Coverage Level $\times$ APH $\times$ Base Price
      \item You are already hedged against price declines
    \end{itemize}
  \end{block}

  \vspace{1em}
  \begin{alertblock}{Important Note}
    If you take a bullish or bearish position when already protected, you are a \highlight{speculator} at that point---not a hedger
  \end{alertblock}
\end{frame}

\begin{frame}{When Prices Rise After Base Price is Set}
  \begin{block}{The Interesting Marketing Question}
    When price rises after the Base Price is set (first of March):
    \begin{itemize}
      \item Producer is exposed to price uncertainty to both upside and downside
      \item However, with Revenue Protection, revenue is guaranteed at the harvest price
      \item \textit{Unless} price falls below the Base Price
    \end{itemize}
  \end{block}

  \vspace{1em}
  \begin{exampleblock}{Key Decision}
    Should a producer lock in a high price that has risen above the Base Price?
  \end{exampleblock}
\end{frame}

\begin{frame}{Marketing Scenario Setup}
  \begin{columns}
    \begin{column}{0.5\textwidth}
      \begin{block}{Assumptions}
        \begin{itemize}
          \item Base Price: \$5.00
          \item BreakEven Price: \$4.90
          \item Base Price $>$ BreakEven (favorable)
          \item July Dec Corn Futures: \$5.35
          \item Substantial rise above Base Price
        \end{itemize}
      \end{block}
    \end{column}

    \begin{column}{0.5\textwidth}
      \begin{block}{Policy Parameters}
        \begin{itemize}
          \item APH: 200 bushels/acre
          \item Coverage Level: 85\%
          \item Revenue Protection policy
        \end{itemize}
      \end{block}
    \end{column}
  \end{columns}

  \vspace{1em}
  \begin{exampleblock}{Question}
    Should a producer lock in the \$5.35 price in July?
  \end{exampleblock}
\end{frame}

\begin{frame}{Price Scenarios: July to October}
  \begin{center}
    \includegraphics[width=0.8\textwidth]{../../assets/cropins_ex.png}
  \end{center}
\end{frame}

\begin{frame}{Marketing Choices to Analyze}
  \begin{block}{Price Scenarios (July $\rightarrow$ October)}
    \begin{enumerate}
      \item Price rises to \highlight{\$5.60}
      \item Price stays flat at \highlight{\$5.35}
      \item Price falls to \highlight{\$5.10}
    \end{enumerate}
  \end{block}

  \vspace{0.5em}
  \begin{block}{Marketing Strategies}
    \begin{itemize}
      \item Selling futures
      \item Contracting for harvest delivery (assume zero basis)
    \end{itemize}
  \end{block}

  \vspace{0.5em}
  \begin{block}{Yield Scenarios}
    \begin{itemize}
      \item 100\% of APH (200 bu/acre)
      \item 80\% of APH (160 bu/acre)
    \end{itemize}
  \end{block}
\end{frame}

\begin{frame}{Interactive Analysis Tool}
  \begin{block}{Explore the Marketing Scenarios}
    You can explore the moving parts yourself using the interactive Google Sheets tool:

    \vspace{1em}
    \centering
    \url{https://docs.google.com/spreadsheets/d/1rbne8odUljxIuIYP3IRiWyzaQDs83-mjOkuaaDWxb3w/edit?usp=sharing}
  \end{block}

  \vspace{1em}
  \begin{exampleblock}{What You Can Analyze}
    \begin{itemize}
      \item Different price scenarios
      \item Different yield outcomes
      \item Impact of marketing decisions on final revenue
      \item Interaction between crop insurance and marketing strategies
    \end{itemize}
  \end{exampleblock}
\end{frame}

%% =============================================================================
%% SUMMARY
%% =============================================================================

\section{Summary}

\begin{frame}{Key Takeaways}
  \begin{enumerate}
    \item \highlight{Crop insurance} is now the dominant federal farm program for risk management (62\% premium subsidy)

    \vspace{0.5em}
    \item Federal involvement is necessary because crop losses are \highlight{highly correlated}---private markets cannot provide adequate liquidity

    \vspace{0.5em}
    \item \highlight{Revenue Protection} (most popular) guarantees:
    \[
      \text{APH} \times \max(\text{Projected Price}, \text{Harvest Price})
    \]

    \vspace{0.5em}
    \item Marketing \highlight{insured bushels} is safer---Revenue Protection provides a floor to buy out contracts if needed

    \vspace{0.5em}
    \item Pre-harvest marketing becomes interesting when prices \highlight{rise above} the Base Price
  \end{enumerate}
\end{frame}

\begin{frame}{Questions for Discussion}
  \begin{enumerate}
    \item Why can't private insurance companies offer crop insurance without federal support?

    \vspace{1em}
    \item What is the key difference between Yield Protection and Revenue Protection?

    \vspace{1em}
    \item When does crop insurance alone fail to provide an optimal marketing plan?

    \vspace{1em}
    \item Why is marketing uninsured bushels particularly risky?
  \end{enumerate}
\end{frame}

%% =============================================================================
%% END OF SLIDES
%% =============================================================================

\end{document}
