\documentclass[aspectratio=169]{beamer}
\usetheme{Madrid}

\usepackage{graphicx}
\usepackage{booktabs}
\usepackage{hyperref}

% Purdue Colors
\definecolor{PurdueGold}{HTML}{CEB888}
\definecolor{PurdueBlack}{HTML}{000000}

% Apply Purdue color theme
\setbeamercolor{palette primary}{bg=PurdueBlack,fg=PurdueGold}
\setbeamercolor{palette secondary}{bg=PurdueGold,fg=PurdueBlack}
\setbeamercolor{palette tertiary}{bg=PurdueBlack,fg=PurdueGold}
\setbeamercolor{palette quaternary}{bg=PurdueGold,fg=PurdueBlack}
\setbeamercolor{structure}{fg=PurdueBlack}
\setbeamercolor{section in toc}{fg=PurdueBlack}
\setbeamercolor{title}{fg=PurdueGold,bg=PurdueBlack}
\setbeamercolor{frametitle}{fg=PurdueGold,bg=PurdueBlack}
\setbeamercolor{block title}{bg=PurdueBlack,fg=PurdueGold}
\setbeamercolor{block body}{bg=PurdueGold!20,fg=PurdueBlack}
\setbeamercolor{item}{fg=PurdueBlack}
\setbeamercolor{subitem}{fg=PurdueBlack}
\setbeamercolor{subsubitem}{fg=PurdueBlack}

\title{Commodity Price Analysis and Forecasting}
\subtitle{Introduction to Commodity Markets}
\author{Mindy Mallory}
\date{}

\begin{document}

\begin{frame}
\titlepage
\end{frame}

\begin{frame}{What is a Commodity?}
A commodity is a good that can be supplied without qualitative differences.

\vspace{1em}

\begin{itemize}
    \item Fully or partially fungible
    \item Market treats a unit the same no matter who produced it or where
    \item Example: A bushel of wheat is regarded as a bushel of wheat everywhere
\end{itemize}

\vspace{1em}

Contrast with differentiated goods where branding and quality matter. Try to find someone indifferent between iPhone and Android!
\end{frame}

\begin{frame}{Fungibility in Practice}
\textbf{Grain Elevator Example:}
\begin{itemize}
    \item Farmers bring grain to an elevator at harvest
    \item Sometimes sell outright, sometimes pay for storage
    \item When farmer retrieves grain from storage, does he get the exact same kernels?
    \item No---elevator gives back the same amount and quality
    \item Farmer is happy because wheat is fungible
\end{itemize}
\end{frame}

\begin{frame}{Commodity Markets}
Since commodities are fungible:
\begin{itemize}
    \item Prices determined by the entire (often global) market
    \item Tend to be basic resources:
    \begin{itemize}
        \item Agricultural and food products
        \item Metals
        \item Energy
        \item Fibers
    \end{itemize}
    \item Fungibility enables trading in centralized spot and futures markets
\end{itemize}
\end{frame}

%------------------------------------------------------------
% TRANSFORMATION
%------------------------------------------------------------

\begin{frame}{}
\centering
\Huge Transformation Over\\Space, Time, and Form
\end{frame}

\begin{frame}{Transformation Over Space}
\begin{itemize}
    \item Production of commodities is often concentrated in specific geographic locations
    \item Consumption is usually dispersed
    \item For traders to move a commodity from one location to another, a certain pattern of prices must prevail
    \item Traders must be able to make a profit (or at least break even)
\end{itemize}
\end{frame}

\begin{frame}{Transformation Over Time}
\begin{itemize}
    \item What prices are required to provide incentive to store a commodity for later use?
    \item Example: Grain is produced once per year (in the U.S.), but consumption occurs all year
    \item Prices through time give incentive for stockholders to bring grain to market or hold on longer
    \item Market coordinates just the right amount of grain to be stored through time
\end{itemize}
\end{frame}

\begin{frame}{Transformation Over Form}
Commodities can be transformed into completely different goods.

\vspace{1em}

\textbf{Creating new commodities:}
\begin{itemize}
    \item Soybeans $\rightarrow$ soybean oil + soybean meal
\end{itemize}

\vspace{1em}

\textbf{Creating differentiated products:}
\begin{itemize}
    \item Feeder cattle $\rightarrow$ live cattle $\rightarrow$ different cuts of meat
    \item Green coffee beans $\rightarrow$ roasted, ground, brewed coffee (Starbucks)
\end{itemize}
\end{frame}

%------------------------------------------------------------
% STORABILITY
%------------------------------------------------------------

\begin{frame}{}
\centering
\Huge Storable and Non-Storable\\Commodities
\end{frame}

\begin{frame}{Storable Commodities}
Can be stored for long periods of time:
\begin{itemize}
    \item Corn
    \item Soybeans
    \item Wheat
    \item Peanuts
    \item Crude Oil
    \item Natural Gas
\end{itemize}
\end{frame}

\begin{frame}{Non-Storable Commodities}
Highly perishable or otherwise non-storable:
\begin{itemize}
    \item Hogs
    \item Cattle
    \item Milk
    \item Potatoes
    \item Apples
    \item Tomatoes
    \item Electricity
\end{itemize}
\end{frame}

\begin{frame}{Implications of Storability}
\textbf{Storable commodities:}
\begin{itemize}
    \item Can be stored from one period to the next
    \item Prices in one period must be related to prices in another period
    \item Stockholders constantly calculate: sell now or later?
\end{itemize}

\vspace{1em}

\textbf{Non-storable commodities:}
\begin{itemize}
    \item Prices can only be affected by current supply
    \item Past supply cannot be brought forward
\end{itemize}
\end{frame}

%------------------------------------------------------------
% IMPORTANCE
%------------------------------------------------------------

\begin{frame}{}
\centering
\Huge Why Commodity Prices Matter
\end{frame}

\begin{frame}{Economic and Political Importance}
Commodity prices are important both economically and politically in almost all countries.

\vspace{1em}

\begin{itemize}
    \item Strongly influence farm income (can be quite volatile year-to-year)
    \item U.S. has long history of policies aimed at smoothing price and income volatility:
    \begin{itemize}
        \item Price supports
        \item Revenue supports
        \item Subsidized crop insurance programs
    \end{itemize}
\end{itemize}
\end{frame}

\begin{frame}{Global Importance}
\begin{itemize}
    \item Some countries' economies rely heavily on commodity exports
    \begin{itemize}
        \item Economic growth subject to commodity price volatility
    \end{itemize}
    \item In developing countries, large share of population engages in agricultural production
    \begin{itemize}
        \item Commodity prices determine bulk of their income
        \item Incomes of the poor is a primary concern
    \end{itemize}
\end{itemize}
\end{frame}

\begin{frame}{Forecasting in Business}
Companies exposed to price volatility spend considerable resources analyzing prices:
\begin{itemize}
    \item ADM
    \item Cargill
    \item Caterpillar
    \item ConAgra
    \item Kraft
    \item Weyerhaeuser
\end{itemize}

\vspace{1em}

Consistent employment opportunities for students trained in price analysis, forecasting, and risk management.
\end{frame}

%------------------------------------------------------------
% ANALYSIS VS FORECASTING
%------------------------------------------------------------

\begin{frame}{}
\centering
\Huge Price Analysis vs.\ Forecasting
\end{frame}

\begin{frame}{Price Analysis}
\textbf{Backward looking}

\vspace{1em}

Goals:
\begin{itemize}
    \item Understand the complex array of forces that influence commodity price levels and behavior
    \item Aid in understanding performance of commodity markets
    \item Aid in development of policy
    \item Key component of policy analysis
\end{itemize}
\end{frame}

\begin{frame}{Price Forecasting}
\textbf{Forward looking}

\vspace{1em}

Goals:
\begin{itemize}
    \item Reliably and accurately forecast future price levels
    \item Forecasts used in:
    \begin{itemize}
        \item Marketing strategies
        \item Risk management
        \item Speculative strategies
    \end{itemize}
\end{itemize}
\end{frame}

%------------------------------------------------------------
% FORECASTING BASICS
%------------------------------------------------------------

\begin{frame}{}
\centering
\Huge Forecasting Basics
\end{frame}

\begin{frame}{Forecasting Basics: Key Principle}
\begin{center}
\Large All meaningful forecasts guide decisions.
\end{center}

\vspace{1em}

An awareness of the nature of the decisions will impact the design, use, and evaluation of the forecasting process.
\end{frame}

\begin{frame}{Forms of Forecast Statements}
\textbf{Directional forecast:}\\
Fed steer prices for Q1 2016 will be down compared to the same quarter last year.

\vspace{0.5em}

\textbf{Simple point forecast:}\\
Fed steer prices for Q1 2016 = \$150/cwt.

\vspace{0.5em}

\textbf{Interval forecast:}\\
Fed steer prices for Q1 2016 = \$140--\$160/cwt.
\end{frame}

\begin{frame}{Forms of Forecast Statements (cont.)}
\textbf{Confidence interval forecast:}\\
We are 80\% confident that fed steer prices for Q1 2016 will be between \$140--\$160/cwt.

\vspace{1em}

\textbf{Density forecast:}\\
Provides entire probability distribution of forecast price.
\end{frame}

\begin{frame}{Forecast Horizon}
Forecast horizon = number of periods between today and the forecast date.

\vspace{1em}

\textbf{With monthly data:}
\begin{itemize}
    \item 1-step ahead = one month beyond current month
    \item 2-step ahead = two months beyond current month
    \item $h$-step ahead = $h$ months beyond current month
\end{itemize}

\vspace{1em}

\textbf{Crop market forecasting:}
\begin{itemize}
    \item Typical unit of time is a ``marketing year''
    \item Forecasts typically updated monthly
\end{itemize}
\end{frame}

\begin{frame}{Parsimony Principle (Occam's Razor)}
\begin{quote}
Among competing hypotheses that predict equally well, the one with the fewest assumptions should be selected.
\end{quote}

\vspace{1em}

\textbf{Other things equal, simple approaches are preferred.}
\end{frame}

\begin{frame}{Why Simple Models?}
\begin{itemize}
    \item Simple approaches tend to work best in real world applications (decades of experience and research)
    \item Simpler models can be estimated more precisely
    \item Unusual behavior and outcomes more easily spotted
    \item Easier to communicate $\rightarrow$ more likely to be used by decision-makers
    \item Lessens chances of data mining problems
\end{itemize}

\vspace{1em}

If a complex model is tailored to fit historical data very well but does not capture the true data process, forecasts will perform poorly.
\end{frame}

\begin{frame}{Two ``Simple'' Forecasting Methods}
\textbf{1. Fundamental Analysis}
\begin{itemize}
    \item Use of economic models and data on production, consumption, income, etc.
    \item Balance sheet analysis (Chapter 3)
\end{itemize}

\vspace{1em}

\textbf{2. Reduced Form Time-Series Econometrics}
\begin{itemize}
    \item Statistical econometric models
    \item Minimal inputs beyond a few recent prices
\end{itemize}

\vspace{1em}

\textit{Not covered here: Technical analysis (use of past price patterns to predict future movement)}
\end{frame}

%------------------------------------------------------------
% PRODUCTION CYCLES AND LONG TERM TRENDS
%------------------------------------------------------------

\begin{frame}{Commodity Production Cycles}
Production of agricultural commodities is bound by biological traits of the life cycle.

\vspace{1em}

Forecasting prices requires:
\begin{itemize}
    \item Awareness of key seasons
    \item Understanding of problems that can arise during each phase of the life cycle
\end{itemize}
\end{frame}

\begin{frame}{Long Term Trends}
\centering
\includegraphics[width=0.75\textwidth]{../../assets/PrimerforGrain_CornPrices.png}

\tiny{Monthly prices received by farmers in the U.S., 1908--2015}
\end{frame}

\begin{frame}{Three Price Regimes in Corn}
Irwin and Good (2016) identify three distinct periods:

\vspace{1em}

\begin{enumerate}
    \item \textbf{Pre-1973:} Stable, low prices
    \item \textbf{1973--2005:} Higher plateau
    \item \textbf{2006--present:} New, higher era (most volatile)
\end{enumerate}

\vspace{1em}

Clear run-up in prices in the 1970s and again around 2006--2007.

What caused these seemingly permanent price hikes?
\end{frame}

\begin{frame}{Summary}
\begin{itemize}
    \item Commodities are fungible goods traded in centralized markets
    \item Transformation over space, time, and form drives price relationships
    \item Storability has profound implications for price dynamics
    \item Price analysis (backward) vs.\ forecasting (forward)
    \item Parsimony: simple models often work best
    \item Long-term corn prices show distinct ``eras''
\end{itemize}
\end{frame}

\end{document}
